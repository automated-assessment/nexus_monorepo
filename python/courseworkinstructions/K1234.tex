\documentclass{article}
\usepackage{graphicx}
\usepackage{enumerate}
\usepackage{verbatim}
\usepackage{bbm}

\newcommand{\diff}{\mathrm{d}}
\newcommand{\ed}{\mathrm{d}}
\newcommand{\ol}{\overline}

\begin{document}

\section{Student Details}
\begin{tabular}{ll}
Name: & Bourne, Jason\\ 
K-Number: & K1234\\ 
Student Number: & 5678\\ 
\end{tabular}
\section{Question for student K1234}

Write a function \verb=answerProblem= which returns two values \verb=part1= and
\verb=part2= where these values are a Monte Carlo approximation to the integrals
\[
\mathtt{part1} \approx \int_0^\frac{\pi}{2} \cos(x) \ed x
\]
and
\[
\mathtt{part2} \approx \int_0^1 \exp(x) \ed x.
\]
Use 1000000 samples for your calculation.

\section{Answer}

Let $u_1$, $u_2$, \ldots $u_n$ be uniformly distributed on the interval
$[0,1]$ then we can estimate the integral of $f(x)$ using the formula
\[
\frac{(b-a)}{n} \sum_{i=1}^n f( a + (b-a)u_i ).
\]

Implementing this in MATLAB we find that
\[
\mathtt{part1} = \int_0^\frac{\pi}{2} \cos(x) \ed x \approx 1.0002.
\]
\[
\mathtt{part2} = \int_0^1 \exp(x) \ed x \approx 1.7817.
\]

I tested the code by writing a general Monte Carlo integrator that worked
for any $f$. I then tested this general function using the analytic formula
\[
\int_0^1 x^2 \ed x = \frac{1}{3} \approx 0.3333.
\]


\end{document}